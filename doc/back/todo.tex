% !TeX root = ../whu-thesis-doc.tex

\section{To-do}

\subsection{大方向}

\begin{itemize}
  \item 写一些用户注意事项
    \begin{itemize}
      \item 层级(见本科规范P11的 |(九)正文|)
      \item \href{https://github.com/whutug/whu-thesis/issues/63}{\#63}:参考文献中出现重音时的处理和 backend 有关
    \end{itemize}
  \item 尽可能复刻黄老师的模版效果并优化代码
  \item 参考外语学院的 word 模版,提供调整接口
  \item 参考 BenjaminHb 对学术硕士的需求的调整
  \item 尝试用 \LaTeX{} 复刻硕士开题报告的表格
  \item 整理 github 上的 issues 和 discussions
  \item 不放过规范的每一个细节,将一些细节给用户写清楚。基于不同的规范来写三个版本(本硕博)的模版示例
  \item 不同学院的差异通过自主定义的配置文件实现(配合 issues 或者手册等让用户知道怎么改)
  \item 学习 \cls{fduthesis}、 \cls{thuthesis}、\cls{ustcthesis}、\cls{ucasthesis}、 \cls{sjtuthesis}、\cls{gdutthesis}、\cls{xdupgthesis} 等模版的细节处理
  \item 增加一些宏包使用样例,比如 \pkg{tabularray}、\pkg{fixdif}、\pkg{listings} 等
  \item 学习 \cls{thuthesis} 等模版的程序文件来处理以及 github 的 actions
  \item 模版完善后移植到 gitee 上
\end{itemize}

\subsection{细节问题}
\begin{itemize}
  \item 实现诸如 \tn{AtEndPreamble} 之类的钩子,取消对 \pkg{etoolbox} 的依赖
\end{itemize}


\subsection{待开发}

\subsubsection{本科}

\begin{itemize}
  \item 封面接口
\end{itemize}


\subsubsection{硕士}

\begin{itemize}
  \item 封面接口
\end{itemize}


\subsubsection{博士}

\begin{itemize}
  \item 封面接口
  \item 攻博期间发表的与学位论文相关的科研成果目录(参考 \cls{CCNUthesis})
\end{itemize}



\subsection{issues 问题解决 To-do}

\begin{itemize}
  \item 图表目录的 label 宽度问题(参考西电论文模版的设置)
  \item \href{https://github.com/whutug/whu-thesis/issues/159}{\#159}:增加页眉的上方横线的控制接口
  \item \href{https://github.com/whutug/whu-thesis/issues/153}{\#153}、\href{https://github.com/whutug/whu-thesis/issues/142}{\#142}、\href{https://github.com/whutug/whu-thesis/issues/125}{\#125}、\href{https://github.com/whutug/whu-thesis/issues/120}{\#120}、\href{https://github.com/whutug/whu-thesis/issues/94}{\#94}、\href{https://github.com/whutug/whu-thesis/issues/8}{\#8}、\href{https://github.com/whutug/whu-thesis/discussions/123}{\#123}:数学字体的配置,注意 \tn{mathscr}、\tn{mathcal} 和 \tn{mathbb} 的效果(参考 \cls{fduthesis} 的 \href{https://github.com/stone-zeng/fduthesis/discussions/270}{\#270}
  \item \href{https://github.com/whutug/whu-thesis/issues/151}{\#151}:参考文献的排序问题(提供接口选择)
  \item \href{https://github.com/whutug/whu-thesis/issues/148}{issue \#148}、\href{https://github.com/whutug/whu-thesis/discussions/147}{discussion \#147}:本科模版增加接口使得中英文摘要和目录页的页码为罗马数字
  \item \href{https://github.com/whutug/whu-thesis/issues/142}{\#142}:算法的配置问题、图表题的参考文献引用问题
  \item \href{https://github.com/whutug/whu-thesis/issues/132}{\#132}、\href{https://github.com/whutug/whu-thesis/issues/24}{\#24}:增加超链接的框接口(参考 \cls{fduthesis})
  \item \href{https://github.com/whutug/whu-thesis/issues/131}{\#131}:处理浮动体和文字之间的间距(参考 \href{https://github.com/sikouhjw/gdutthesis/blob/c3cc2de5bfa47f7bf1b88c0884cca60119d6fd82/gdutthesis.cls#L1106-L1132}{\cls{gdutthesis} 的补丁})
  \item \href{https://github.com/whutug/whu-thesis/issues/130}{\#130}:配置好 A4 纸张
  \item \href{https://github.com/whutug/whu-thesis/issues/152}{\#152}、\href{https://github.com/whutug/whu-thesis/issues/117}{\#117}、\href{https://github.com/whutug/whu-thesis/issues/44}{\#44}:硕博模版上传图书馆需要去掉所有空白页(关键点在于页码、页眉的处理)处理好空白页(给用户提供接口,比如要不要空白页,然后空白页有无页码等)
  \item \href{https://github.com/whutug/whu-thesis/issues/128}{\#128}:个人信息接口需要填写两个的话,要允许有一个缺失
  \item \href{https://github.com/whutug/whu-thesis/issues/92}{\#92}:攻博期间发表的科研成果编号问题(参考 \cls{CCNUthesis}),提供键值让用户选择是否编号(有没有可能通过 \tn{bibentry} 和 \tn{nobibliography} 之类进行处理)
  \item \href{https://github.com/whutug/whu-thesis/issues/85}{\#85}、\href{https://github.com/whutug/whu-thesis/issues/76}{\#76}:处理参考文献的引用命令
  \item \href{https://github.com/whutug/whu-thesis/issues/80}{\#80}:整合 zepinglee 写的本科 bst 到模版中
  \item \href{https://github.com/whutug/whu-thesis/issues/74}{\#74}、\href{https://github.com/whutug/whu-thesis/issues/72}{\#72}、\href{https://github.com/whutug/whu-thesis/issues/56}{\#56}、\href{https://github.com/whutug/whu-thesis/issues/39}{\#39}、\href{https://github.com/whutug/whu-thesis/issues/30}{\#30}:中文字体的配置
  \item \href{https://github.com/whutug/whu-thesis/issues/71}{\#71}:重定义 \tn{figureautorefname} 等名称来使用 \tn{autoref}
  \item \href{https://github.com/whutug/whu-thesis/issues/70}{\#70}:配置好 \pkg{hyperref} 宏包
  \item \href{https://github.com/whutug/whu-thesis/issues/67}{\#67}:本硕如果有硕士期间研究成果要添加,可以放在附录
  \item \href{https://github.com/whutug/whu-thesis/issues/61}{\#61}、\href{https://github.com/whutug/whu-thesis/issues/53}{\#53}、\href{https://github.com/whutug/whu-thesis/issues/46}{\#46}、\href{https://github.com/whutug/whu-thesis/issues/36}{\#36}、\href{https://github.com/whutug/whu-thesis/issues/4}{\#4}:参考文献的样式问题
  \item \href{https://github.com/whutug/whu-thesis/issues/135}{\#135}、\href{https://github.com/whutug/whu-thesis/issues/66}{\#66}、\href{https://github.com/whutug/whu-thesis/issues/51}{\#51}、\href{https://github.com/whutug/whu-thesis/issues/50}{\#50}、\href{https://github.com/whutug/whu-thesis/issues/3}{\#3}、\href{https://github.com/whutug/whu-thesis/discussions/103}{discussion \#103}:增加接口控制的目录缩进(subsub 的可以默认设置为 sub + 2em)
  \item \href{https://github.com/whutug/whu-thesis/issues/55}{\#55}:测试图表标题内的参考文献引用
  \item \href{https://github.com/whutug/whu-thesis/issues/52}{\#52}:兼容 \tn{bm} 宏包?
  \item \href{https://github.com/whutug/whu-thesis/issues/47}{\#47}:title 中含有公式的测试
  \item \href{https://github.com/whutug/whu-thesis/issues/38}{\#38}:\pkg{biblatex} 与钩子的问题
  \item \href{https://github.com/whutug/whu-thesis/issues/10}{\#10}:\pkg{listings} 配置?
  \item \href{https://github.com/whutug/whu-thesis/issues/1}{\#1}:提供用户几种编辑器或 overleaf 的编译方式示例
\end{itemize}


\subsection{discussions 问题解决 To-do}

\begin{itemize}
  \item \href{https://github.com/whutug/whu-thesis/discussions/164}{\#164}:增加联合导师的接口
  \item \href{https://github.com/whutug/whu-thesis/discussions/163}{\#163}:\tn{section} 中引用测试
  \item \href{https://github.com/whutug/whu-thesis/discussions/160}{\#160}:超链接的颜色接口
  \item \href{https://github.com/whutug/whu-thesis/discussions/160}{\#160}:参考文献字号问题
  \item \href{https://github.com/whutug/whu-thesis/discussions/157}{\#157}:公式和文本的数字字体测试
  \item \href{https://github.com/whutug/whu-thesis/discussions/155}{\#155}:参考文献的同名问题
  \item \href{https://github.com/whutug/whu-thesis/discussions/154}{\#154}:文献只引用标题问题
  \item \href{https://github.com/whutug/whu-thesis/discussions/136}{\#136}:参考文献的溢出问题
  \item \href{https://github.com/whutug/whu-thesis/discussions/134}{\#134}:参考文献的“标准”类的引用问题
  \item \href{https://github.com/whutug/whu-thesis/discussions/124}{\#124}:设计硕博的目录层级接口
  \item \href{https://github.com/whutug/whu-thesis/discussions/111}{\#111}:脚注的序号和文字的距离调整
  \item \href{https://github.com/whutug/whu-thesis/discussions/105}{\#105}:宏包依赖(尽可能减少宏包依赖,而增加宏包载入的配置)
  \item \href{https://github.com/whutug/whu-thesis/discussions/104}{\#104}:提供标点符号的接口(开明等)
  \item \href{https://github.com/whutug/whu-thesis/discussions/103}{\#103}:参考文献中的破折号(页码等)
  \item \href{https://github.com/whutug/whu-thesis/discussions/100}{\#100}:多 \file{.bib} 文件
  \item \href{https://github.com/whutug/whu-thesis/discussions/98}{\#98}:word 模版制作?
  \item \href{https://github.com/whutug/whu-thesis/discussions/89}{\#89}:重新设计图表索引(目录)接口(去掉键值而改为命令形式)
  \item \href{https://github.com/whutug/whu-thesis/discussions/78}{\#78}:增加版本检测
  \item \href{https://github.com/whutug/whu-thesis/discussions/75}{\#75}:参考文献的引用
  \item \href{https://github.com/whutug/whu-thesis/discussions/42}{\#42}:考虑把封面和扉页用 \pkg{xtemplate} 重构使得能够兼容更多学院的要求
  \item \href{https://github.com/whutug/whu-thesis/discussions/43}{\#43}:Docker 镜像、github 的actions
\end{itemize}