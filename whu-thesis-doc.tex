\documentclass{fdudoc}

\usepackage{zhlineskip}
\usepackage{enumitem}
\setlist{nosep}

\newcommand\WhuThesis{\textsc{Whu\-The\-sis}}

\title{\bfseries\WhuThesis:武汉大学毕业论文模板}
\author{whu-thesis}
\date{2020/12/30 \quad v0.4}

\begin{document}
\maketitle
\tableofcontents

\section{概述}
\WhuThesis(\textbf{W}u\textbf{h}an \textbf{U}niversity \LaTeX{} \textbf{Thesis} Template)是为了帮助武汉大学毕业生撰写毕业论文(设计)而编写的 \LaTeX{} 模板,内部使用 \LaTeX3 语言,以适应 \TeX{} 技术发展潮流。

现时段 \WhuThesis 暂时只提供本科生毕业论文(设计)模板。

模板根据《武汉大学本科生毕业论文(设计)书写印制规范》编写,力求合规,简洁,易于实现,用户友好。

与 MS Word 等所见即所得编辑工具不同,使用 \LaTeX{} 工具排版可以将写作与排版过程分离,写作者只需要关心文字的部分,而剩下的排版工作全部交给工具自动完成。

模板的作用在于减少论文写作过程中格式调整的时间。前提是遵守模板的用法,否则即便用了 \WhuThesis 也难以保证输出的论文符合学校规范。

用户如果遇到 bug,或者发现与学校《印制规范》的要求不一致,可以尝试以下办法:
\begin{enumerate}
    \item 阅读学校的\href{https://github.com/mtobeiyf/whu-thesis/files/4638713/default.pdf}{书写印制规范文件},判断是否符合要求;
    \item 前往项目 \href{https://github.com/mtobeiyf/whu-thesis/wiki}{wiki} 查看相关说明;
    \item 将 \TeX{} 发行版和宏包升级到最新,并且将模板升级到 Github 上最新版本,查看问题是否已经修复;
    \item 在 \href{https://github.com/mtobeiyf/whu-thesis/issues}{GitHub Issues 页面}中搜索该问题的关键词;
    \item 提出新的 \href{https://github.com/mtobeiyf/whu-thesis/issues}{issue},并说明系统、\TeX{} 版本、出现的问题等关键信息。
\end{enumerate}

\subsection*{关于本手册}
本手册假定用户已经能处理一般的 \LaTeX{} 文档,并对 \hologo{BibTeX} 有一定了解。如果从未接触过 \TeX{} 和 \LaTeX{},建议先学习相关的基础知识。

本文采用不同字体表示不同内容。无衬线字体表示宏包名称,如 \pkg{xeCJK} 宏包、\cls{ctexbook} 文档类等;等宽字体表示代码或文件名,如 \cs{whusetup} 命令、\env{abstract} 环境、\TeX{} 文档 \file{thesis.tex} 等;带有尖括号的楷体(或西文斜体)表示命令参数,如 \meta{模板选项}、\meta{English title} 等。在使用时,参数两侧的尖括号不必输入。示例代码进行了语法高亮处理,以方便阅读。

在用户手册中,带有蓝色侧边线的为 \LaTeX{} 代码,而带有粉色侧边线的则为命令行代码,请注意区分。模板提供的选项、命令、环境等,均用横线框起,同时给出使用语法和相关说明。

本手册使用 \href{https://github.com/stone-zeng/fduthesis}{\pkg{fduthesis}}\scite{fduthesis} 所附带的 \cls{fdudoc} 文档类编写,在此向 \pkg{fduthesis} 的作者\href{https://github.com/stone-zeng}{曾祥东}先生表示感谢。

\section{使用说明}

\subsection{基本用法}

以下是一份简单的 \TeX{} 文档,它演示了 \WhuThesis 的最基本用法:
\begin{latexexample}[deletetexcs={\documentclass},
        moretexcs={\chapter},morekeywords={\documentclass},
        emph={[2]document}]
    % thesis.tex
    \documentclass{whuthesis}
    \begin{document}
    \chapter{欢迎}
    \section{Welcome to WhuThesis!}
    你好,\LaTeX{}!
    \end{document}
\end{latexexample}

按照 \ref{subsec:编译方式} 小节中的方式编译该文档,您应当得到一篇 5 页的文章。当然,这篇文章的绝大部分都是空白的。

\subsection{编译方式}\label{subsec:编译方式}
本模板只支持 \hologo{XeLaTeX} 或 \hologo{LuaLaTeX} 引擎,其他编译方式会直接报错。为了生成正确的目录、脚注以及交叉引用,您至少需要连续编译两次。

以下代码中,假设您的 \TeX{} 源文件名为 \file{thesis.tex}。使用 \XeLaTeX{} 编译论文,请在命令行中执行
\begin{shellexample}[morekeywords={xelatex,bibtex}]
    xelatex thesis
    bibtex  thesis
    xelatex thesis
    xelatex thesis
\end{shellexample}
或使用 \pkg{latexmk}:
\begin{shellexample}[morekeywords={latexmk},emph={-xelatex}]
    latexmk -xelatex thesis
\end{shellexample}

使用 \LuaLaTeX{} 编译论文,请在命令行中执行
\begin{shellexample}[morekeywords={lualatex,bibtex}]
    lualatex thesis
    bibtex   thesis
    lualatex thesis
    lualatex thesis
\end{shellexample}
或者
\begin{shellexample}[morekeywords={latexmk},emph={-lualatex}]
    latexmk -lualatex thesis
\end{shellexample}

在特殊情况下,可能需要在编译时加入 \opt{-shell-escape} 选项,如
\begin{shellexample}[morekeywords={latexmk},emph={-xelatex,-shell-escape}]
    latexmk -xelatex -shell-escape thesis
\end{shellexample}

\subsection{模板选项}
所谓“模板选项”,指需要在引入文档类的时候指定的选项:
\begin{latexexample}[deletetexcs={\documentclass},
        morekeywords={\documentclass}]
    \documentclass(*\oarg{模板选项}*){whuthesis}
\end{latexexample}

有些模板选项为布尔型,它们只能在 \opt{true} 和 \opt{false} 中取值。对于这些选项,\kvopt{\meta{选项}}{true} 中的“"="\,\opt{true}”可以省略。

\begin{function}{degree}
    \begin{fdusyntax}[emph={[1]type}]
        type = (*<doctor|master|(bachelor)>*)
    \end{fdusyntax}
    选择学位类型。三种选项分别代表博士学位论文、硕士学位论文和本科毕业论文。默认值是 \opt{bachelor},\emph{\WhuThesis 暂不提供博士与硕士论文模板}。
\end{function}

\begin{function}{class}
    \begin{fdusyntax}[emph={[1]class}]
        class = (*<(paper)|design|opening>*)
    \end{fdusyntax}
    选择论文类型。三种选项分别代表毕业论文、毕业设计和开题报告,默认值是 \opt{paper}。\opt{paper} 与 \opt{design} 选项都会加载 \pkg{ctexbook} 文档类,此二选项只影响论文标题显示为毕业论文还是毕业设计,此选项在《印制规范》中并未体现。\opt{opening} 则会加载开题报告模板,此时加载的是 \pkg{ctexart}文档类。
\end{function}

\begin{function}{draft}
    \begin{fdusyntax}[emph={[1]draft}]
        draft = (*<\TFF>*)
    \end{fdusyntax}
    选择是否开启草稿模式,默认关闭。
\end{function}

草稿模式为全局选项,会影响到很多宏包的工作方式。开启之后,主要的变化有:
\begin{itemize}
    \item 把行溢出的盒子显示为黑色方块;
    \item 不实际插入图片,只输出一个占位方框;
    \item 关闭超链接渲染,也不再生成 PDF 书签;
    \item 显示页面边框,使用 \hologo{LuaLaTeX} 编译时,还会加载 \pkg{lua-visual-debug} 宏包。
\end{itemize}

\begin{function}{oneside,twoside}
    指明论文的单双面模式,论文默认为 \opt{twoside},开题报告默认为 \opt{oneside}。该选项会影响每章的开始位置。开题报告没有章一级,因此不受该选项限制。
\end{function}

在双面模式(\opt{twoside})下,按照通常的排版惯例,每章应只从奇数页(在右)开始;而在单页模式(\opt{oneside})下,则可以从任意页面开始。本模板中,郑重声明、中文摘要、目录均视作章,也按相同方式排版。

\begin{function}{punct}
    \begin{fdusyntax}[emph={[1]punct}]
        punct = (*<(quanjiao)|banjiao|kaiming>*)
    \end{fdusyntax}
    选择标点样式。三个选项分别为“全角”“半角”“开明”式标点。此选项来自于 \pkg{ctex} 宏包,具体可以查看 \pkg{ctex} 宏包文档获得详细说明。
\end{function}

\subsection{参数设置}
\begin{function}{\whusetup}
    \begin{fdusyntax}[morekeywords={\whusetup}]
        \whusetup(*\marg{键值列表}*)
    \end{fdusyntax}
    本模板提供了一系列选项,可由您自行配置。载入文档类之后,以下所有选项均可通过统一的命令 \cs{whusetup} 来设置。
\end{function}

\cs{whusetup} 的参数是一组由(英文)逗号隔开的选项列表,列表中的选项通常是 \kvopt{\meta{key}}{\meta{value}} 的形式。部分选项的 \meta{value} 可以省略。对于同一项,后面的设置将会覆盖前面的设置。在下文的说明中,将用\textbf{粗体}表示默认值。

\cs{whusetup} 采用 \LaTeX3 风格的键值设置,支持不同类型以及多种层次的选项设定。键值列表中,“"="”左右的空格不影响设置;但需注意,参数列表中不可以出现空行。

与模板选项相同,布尔型的参数可以省略 \kvopt{\meta{选项}}{true}中的“"="\,\opt{true}”。

另有一些选项包含子选项,它们可以按如下两种等价方式来设定:
\begin{latexexample}[morekeywords={\whusetup},
        emph={[1]option-a,sub-option-a,option-b,sub-option-b}]
    \whusetup{
        option-a = {
                sub-option-a = value-a
            },
        option-b = {
                sub-option-b = value-b
            }
    }
\end{latexexample}
或者
\begin{latexexample}[morekeywords={\whusetup},
        emph={[1]option-a,sub-option-a,option-b,sub-option-b}]
    \whusetup{
        option-a/sub-option-a = value-a,
        option-b/sub-option-b = value-b
    }
\end{latexexample}

注意 “"/"” 的前后均不可以出现空白字符。

\subsubsection{信息录入} \label{subsubsec:信息录入}
\begin{function}{info}
    \begin{fdusyntax}[emph={[1]info}]
        info = (*\marg{键值列表}*)
        info/(*\meta{key}*) = (*\meta{value}*)
    \end{fdusyntax}
    该选项包含许多子项目,用于录入论文信息。具体内容见下。以下带“"*"”的项目表示对应的英文字段。
\end{function}

\begin{function}{info/title}
    \begin{fdusyntax}[emph={[1]title}]
        title = (*\marg{标题}*)
    \end{fdusyntax}
    论文标题。默认会在约 19 个汉字字宽处强制断行,但为了语义的连贯以及排版的美观,如果您的标题长于一行,建议使用“"\\"”手动断行。
\end{function}

\begin{function}{info/author}
    \begin{fdusyntax}[emph={[1]author}]
        author = (*\marg{姓名}*)
    \end{fdusyntax}
    作者姓名。
\end{function}

\begin{function}{info/student-number}
    \begin{fdusyntax}[emph={[1]student-number}]
        student-number = (*\marg{数字}*)
    \end{fdusyntax}
    作者学号,共 18 位。
\end{function}

\begin{function}{info/school}
    \begin{fdusyntax}[emph={[1]school}]
        school = (*\marg{名称}*)
    \end{fdusyntax}
    院(系)名称。
\end{function}

\begin{function}{info/major}
    \begin{fdusyntax}[emph={[1]major}]
        major = (*\marg{名称}*)
    \end{fdusyntax}
    专业名称。
\end{function}

\begin{function}{info/advisor}
    \begin{fdusyntax}[emph={[1]advisor}]
        advisor = (*\marg{姓名 , 职称}*)
    \end{fdusyntax}
    指导教师姓名与职称。
\end{function}

\begin{function}{info/date}
    \begin{fdusyntax}[emph={[1]date}]
        date = (*\marg{年数字/月数字}*)
    \end{fdusyntax}
    论文完成日期,默认为汉数字格式的文档编译年月。
\end{function}

\begin{function}{info/keywords,info/keywords*}
    \begin{fdusyntax}[emph={[1]keywords,keywords*}]
        keywords  = (*\marg{中文关键词}*)
        keywords* = (*\marg{英文关键词}*)
    \end{fdusyntax}
    关键词列表。各关键字之间需使用英文逗号隔开。为防止歧义,可以用分组括号“"{...}"”把各字段括起来。
\end{function}

\subsubsection{论文格式} \label{subsubsec:论文格式}

\begin{function}{style}
    \begin{fdusyntax}[emph={[1]style}]
        style = (*\marg{键值列表}*)
        style/(*\meta{key}*) = (*\meta{value}*)
    \end{fdusyntax}
    该选项包含许多子项目,用于设置论文格式。具体内容见下。
\end{function}

\begin{function}{style/bib-file}
    \begin{fdusyntax}[emph={[1]bib-file}]
        bib-file = (*\marg{文件}*)
    \end{fdusyntax}
    参考文献数据源。可以是单个文件,也可以是用英文逗号隔开的一组文件。
\end{function}

\begin{function}{style/graphics-path}
    \begin{fdusyntax}[emph={[1]graphics-path}]
        graphics-path = (*\marg{路径}*)
    \end{fdusyntax}
    用于指定图片文件路径,默认为空。
\end{function}

\subsection{正文编写}

\begin{function}{abstract,abstract*}
    \begin{fdusyntax}[emph={[2]abstract,abstract*}]
        \begin{abstract}
            (*\meta{中文摘要}*)
        \end{abstract}

        \begin{abstract*}
            (*\meta{英文摘要}*)
        \end{abstract*}
    \end{fdusyntax}
    摘要。不带星号和带星号的版本分别用来输入中文摘要和英文摘要。英文摘要结束后会自动打印目录。
\end{function}

\begin{function}{\makebibliography}
    打印参考文献列表。参考文献数据源由 \ref{doc/function//style/bib-file} 章的 \opt{style/bib-file} 指定。
\end{function}

\begin{function}{acknowledgements}
    \begin{fdusyntax}[emph={[2]acknowledgements}]
        \begin{acknowledgements}
            (*\meta{致谢}*)
        \end{acknowledgements}
    \end{fdusyntax}
    致谢。致谢环境结束后会自动进入附录环境。附录环境下,"chapter" 计数器会重新计数并以大写字母格式输出。
\end{function}

\section{宏包依赖}
使用不同编译方式、指定不同选项,会导致宏包依赖情况有所不同。
具体如下:
\begin{itemize}
    \item 在任何情况下,本模板都会\emph{显式}调用以下宏包(或文档类):
          \begin{description}
              \item[\pkg{expl3}、\pkg{xparse}、\pkg{xtemplate} 和 \pkg{l3keys2e}] 用于构建 \LaTeX3 编程环境 \scite{source3}。它们分属 \pkg{l3kernel} 和 \pkg{l3packages} 宏集。
              \item[\cls{ctex文档类}] 提供中文排版的通用框架,属于 \CTeX{} 宏集\scite{CTeX}。
              \item[\pkg{amsmath}、\pkg{amssymb} 与 \pkg{amsthm}] 对 \LaTeX{} 的数学排版功能进行了全面扩展,并提供定理类环境定制功能。属于 \AmSLaTeX{} 套件。
              \item[\pkg{mathtools}] 是 \pkg{amsmath} 的扩充,修正了 \pkg{amsmath} 的 bug,并提供了更多数学排版功能。
              \item[\pkg{ti\emph{k}z}、\pkg{pgfplots}] 提供绘图(如圆圈数字)支持。
              \item[\pkg{ulem}] 提供绘制下划线功能。
              \item[\pkg{siunitx}、\pkg{physics}] 提供方便的单位、物理符号输入支持。
              \item[\pkg{algorithm2e}] 用于提供编写算法/伪代码的支持。
              \item[\pkg{geometry}] 用于调整页面尺寸。
              \item[\pkg{fancyhdr}] 处理页眉页脚。
              \item[\pkg{tocloft}] 调整目录样式。
              \item[\pkg{subcaption}] 提供子题注支持。
              \item[\pkg{caption}] 用于设置题注格式。
              \item[\pkg{graphicx}] 提供图形插入的接口。
              \item[\pkg{tabularx}] 用来创建给定总宽度的表格。
              \item[\pkg{longtable}、\pkg{xltabular}] 提供长表格(允许跨页的表格)与定宽长表格的支持。
              \item[\pkg{booktabs}] 提供三线表支持。
              \item[\pkg{multirow}、\pkg{makecell}] 提供纵向合并、“拆分”单元格功能。
              \item[\pkg{diagbox}] 用于绘制斜线表头。
              \item[\pkg{enumitem}] 用于删除列举环境额外的的纵向间距。
              \item[\pkg{footmisc}] 用于设置脚注序号每面更新。
              \item[\pkg{unicode-math}] 负责处理 Unicode 编码的 OpenType 数学字体。
              \item[\pkg{natbib}] 提供作者—年份引用格式。
              \item[\pkg{hyperref}] 提供交叉引用、超链接、电子书签等功能。
          \end{description}
    \item 设置 \kvopt{class}{opening} 时,\WhuThesis 会调用 \cls{ctexart} 文档类,同时会调用 \pkg{tocbibind} 宏包,对参考文献章节编号。而当 \kvopt{class}{paper} 或 \opt{design} 时,\WhuThesis 会调用 \cls{ctexbook} 文档类。
    \item 使用 \XeLaTeX{} 编译时,\CTeX{} 会调用 \pkg{xeCJK}\scite{xeCJK} 宏包,而使用 \LuaLaTeX{} 编译时,\CTeX{} 会调用 \pkg{luatexja}\scite{LuaTeX-ja} 宏包。不同的编译方式和中文支持方式会在一定程度上影响 \CTeX{} 宏集的行为,如对空格、标点的处理等。一般来说,使用 \XeLaTeX{} 编译时,推荐在中西文间显式地插入一个西文空格,而使用 \LuaLaTeX{} 编译时中西文间不插入空格。
    \item \opt{draft} 选项开启时,如果使用 \LuaLaTeX{} 编译,会载入 \pkg{lua-visual-debug} 宏包。
\end{itemize}

这里只列出了本模板直接调用的宏包。这些宏包自身的调用情况,此处不再具体展开。如有需要,请参阅相关文档。

\begin{thebibliography}{99}
    \newcommand\urlprefix{\newline\hspace*{\fill}}
    \let\OldUrl=\url
    \renewcommand\url[2][]{{\small\textit{#1}~\OldUrl{#2}}}
    \newcommand\CTANurl[2][]{{\small\textit{#1}~\href{http://mirror.ctan.org/#2}{\ttfamily CTAN://#2}}}

    \bibitem{source2e}
    \textsc{Braams J}, \textsc{Carlisle D}, \textsc{Jeffrey A}, et al.
    \newblock \textit{The \LaTeXe{} Sources} [CP/OL].
    \newblock (2020-10-01)
    \urlprefix\url{https://ctan.org/pkg/latex}
    \urlprefix\CTANurl[源代码:]{macros/latex/base/source2e.pdf}

    \bibitem{source3}
    \textsc{The \LaTeX3 Project}.
    \newblock \textit{The \LaTeX3 Sources} [CP/OL].
    \newblock (2020-12-07)
    \urlprefix\url{https://ctan.org/pkg/l3kernel}
    \urlprefix\CTANurl[源代码:]{macros/latex/contrib/l3kernel/source3.pdf}

    \bibitem{CTeX}
    \textsc{CTEX.ORG}.
    \newblock \textit{\CTeX{} 宏集手册} [EB/OL].
    \newblock version 2.5.5,
    \newblock (2020-10-19)
    \urlprefix\url{https://ctan.org/pkg/ctex}
    \urlprefix\CTANurl[文档及源代码:]{language/chinese/ctex/ctex.pdf}

    \bibitem{xeCJK}
    \textsc{CTEX.ORG}.
    \newblock \textit{\pkg{xeCJK} 宏包} [EB/OL].
    \newblock version 3.8.6,
    \newblock (2020-10-19)
    \urlprefix\url{https://ctan.org/pkg/xecjk}
    \urlprefix\CTANurl[文档及源代码:]{macros/xetex/latex/xecjk/xeCJK.pdf}

    \bibitem{LuaTeX-ja}
    \LuaTeX-ja プロジェクトチーム.
    \newblock \textit{\LuaTeX-ja パッケージ} [EB/OL].
    \newblock version 20201224.0,
    \newblock (2020-12-24)
    \urlprefix\url{https://ctan.org/pkg/luatexja}
    \urlprefix\CTANurl[文档:]{macros/luatex/generic/luatexja/doc/luatexja-ja.pdf}

    \bibitem{lshort-zh-cn}
    \textsc{Oetiker T}, \textsc{Partl H}, \textsc{Hyna I}, et al.
    \newblock \textit{一份(不太)简短的 \LaTeXe{} 介绍: 或 112 分钟了解 \LaTeXe{}} [EB/OL].
    \newblock \CTeX{} 开发小组, 译.
    \newblock 原版版本 version 6.2, 中文版本 version 6.02,
    \newblock (2020-08-03)
    \urlprefix\url{https://ctan.org/pkg/lshort-zh-cn}
    \urlprefix\CTANurl[文档:]{info/lshort/chinese/lshort-zh-cn.pdf}

    \bibitem{thuthesis}
    清华大学 TUNA 协会.
    \newblock \textit{\textsc{ThuThesis}:清华大学学位论文模板} [EB/OL].
    \newblock version 7.1.0,
    \newblock (2020-10-14)
    \urlprefix\url{https://ctan.org/pkg/thuthesis}
    \urlprefix\CTANurl[文档及源代码:]{macros/latex/contrib/thuthesis/thuthesis.pdf}

    \bibitem{fduthesis}
    曾祥东.
    \newblock \textit{fduthesis: 复旦大学论文模板} [EB/OL].
    \newblock version 0.7e,
    \newblock (2020/08/30)
    \urlprefix\url{https://ctan.org/pkg/fduthesis}
    \urlprefix\CTANurl[文档及源代码:]{macros/latex/contrib/fduthesis/fduthesis-code.pdf}
\end{thebibliography}

\end{document}