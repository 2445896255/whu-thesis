% 致谢页

于武汉大学求学七载,如今到了毕业离别的时间。此间光阴之中,我曾见过雨天纷扬的落樱,看过日落余晖的东湖碧波,走过铺满梧桐落叶的街道,也曾赶过早课的前排,驻足于图书馆的书架前,仰望过实验室窗外的明月稀星。人的一生不过十数个七年,我有幸在珞珈山度过其一,得良师教导,与益友同行,从身无长物成长至略有见识,那些美好的回忆,散落在七年间的日日夜夜中,于今回顾,掇菁撷华,我仍然常常心怀感念。

感谢党和祖国,如果不是生在这片土地上,可能小时候我就无法负担教育的费用,更遑论走入最高学府,我作为一个普通人,能拥有和平和便捷的生活,是因为有许多人已经在负重前行。感谢学校和学院,为学生提供了舒适的学习环境和丰富的学术资源,免费正版软件、各大论文库、前沿讲座……这些材料对我的硕士生涯帮助莫大。

感谢我的导师李石君老师,在我的科研生涯中,李老师总是耐心细致地给予我帮助和指导,我从李老师身上不仅学到了科研的精神和方法,也学到了许多做人做事的道理,在我生病时,李老师也第一时间关心我的状况,在就业上,李老师也给予了我许多真诚的建议和指导。感谢余伟老师,余老师总是不厌其烦的回答我的问题,在学术上给予了我许多宝贵的意见和建议,在我遇到困难时,余老师会及时地关怀我,帮助我调整自己的状态。“师者,所以传道受业解惑也”,两位老师不仅是我学术上的导师,更是我人生的导师。

感谢实验室的同门,硕士三年,大多数时间都待在实验室里,是他们让这段科研旅程变得不再孤单。我的同门都是真诚且优秀的人,从他们身上,我学到了许多宝贵的知识和经验,实验室见证了我们并肩作战的日夜,也听过我们在闲暇时的欢声笑语,我们不仅在学术上互相交流和学习,也在生活中彼此帮助与鼓励,这份同窗情谊永志难忘。

感谢我的朋友们,因为各自学业或工作的缘故我们不常相聚,但我们之间的情谊从未因距离或时间而变淡,有时候一个拥抱就胜过万语千言。感谢他们的陪伴、理解和鼓励,让我的生活里从不缺乏阳光和快乐。

感谢我的家人,一直以来,因为有他们的支持和鼓励,我才有勇气和底气去做想做的事情,在我焦虑的时候,是他们包容我的脾气,无论我走到哪里,他们的爱都陪伴着我,他们都是我最坚实的后盾。


珞珈之风,山高水长。感谢一路上遇到的每位老师和同学,我祝愿老师们福乐安康,祝愿同学朋友们前途光明,在往后的人生路上,我会铭记“自强、弘毅、求是、拓新”的校训,砥砺前行,做一个对国家、对人民、对社会有用的人!