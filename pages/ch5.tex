% chapter5 总结

\chapter{总结与展望}

\section{总结}

随着各国脑科学计划的推进和脑机接口技术的不断发展,运动想象分类领域逐渐吸引了研究者们越来越多的注意,然而,现有的方法仍然存在对数据增强算法和神经科学先验知识具有依赖性,跨被试和不同空间分辨率场景的稳定性较差等问题,需要进一步对方法的通用性、稳定性和简便性进行增强。因此,论文基于特征融合与注意力机制构建了一种端到端的运动想象脑电图分类网络HA-FuseNet,并通过一系列实验证明了其具有利用小规模数据集达到较好的分类精度的能力,同时,在低空间分辨率与跨被试场景中,HA-FuseNet同样具有良好的稳定性,从而验证了模型的有效性。

现将论文的主要研究成果总结如下:

(1) 针对EEG信号特征相对简单,深层语义信息与浅层特征同等重要的特点,构建了多尺度密集连接模块(Dense Inception Module),以提取更为丰富和完整的EEG信号特征。通过反转瓶颈层,促进二维EEG信号的时空特征融合;根据EEG信号的采样频率设计卷积核的大小,以提取对应尺度的时间/频率特征;通过在多个分支应用多个不同尺度的卷积核,扩大特征提取的广度;通过密集连接方式对浅层特征和深层特征加以融合,获取更全面的特征表达;采取集中关注时间维度特征的策略,使用轴向的时间卷积核和空间卷积核,并堆叠相对更深的时间卷积层,以增强模型对不同精度的空间分辨率的适应性;

(2) 针对EEG信号非平稳、信噪比低、冗余信息较多的特点,构建了svSE混合注意力模块,以增强模型对重点数据的关注度。针对MI-EEG信号随时间变化波动明显,呈现出最大类内方差的特点\cite{mane2021fbcnet},svSE模块提出一种方差池化的方法,与平均池化相结合,以均衡且有针对性地对MI-EEG信号特征进行表征;针对EEG信号二维数据时空相关性不强的特点,svSE模块使用轴向注意力机制分别对时空维度进行建模,以适应不同形式的数据分布;

(3) 针对卷积神经网络无法获取全局依赖信息的问题,将LSTM与全局自注意力SCoT模块进行结合(LS-Net),以对EEG信号的长期依赖关系进行建模。SCoT模块采用两阶段计算的方式,首先计算空间域的全局自注意力并对原始数据进行加权,其次计算时空域的全局自注意力,对数据进行二次加权,以获得更全面的校准后的自注意力权重;通过全局依赖信息与局部依赖信息在深度维度的聚合,以在不形成干扰的情况下对局部与全局特征进行利用;

(4) 针对MI-EEG数据规模小,现有方法参数规模大、计算开销高、容易过拟合的问题,对论文所提出的模型HA-FuseNet进行了轻量化,以对小规模数据集进行适应,减少计算开销。HA-FuseNet通过轴向可分离卷积、深度可分离卷积削减参数规模,并提出了SG轻量化卷积模块,将参数规模与FLOPs分别缩减了约75\%和80\%;

(5) 在不同数据集上进行了一系列实验,对论文所提出的HA-FuseNet与多项基准模型的性能进行评估,并对实验结果展开了一系列讨论。在具有3通道的BCI Competition IV Dataset 2B数据集上,被试内实验中,HA-FuseNet的平均准确率和Kappa系数较次优模型EEGInception分别提升了约0.65\%和0.01,准确率标准差优化了约0.05;被试间实验中,HA-FuseNet的平均准确率、Kappa系数和准确率标准差分别比次优模型优化了约1.22\%、0.01和0.61。在22通道的2A数据集上,HA-FuseNet同样取得了最优的表现。

\section{展望}

论文所构建的基于特征融合和注意力机制的HA-FuseNet在运动想象脑电图分类任务上的精度、稳定性与泛化能力有所提升,显现出一定的优势,然而,模型仍然存在一定的进步空间。在论文研究的基础上,可以从以下方向考虑运动想象脑电图分类领域进一步的工作:

(1) 更精细的运动想象:在真实场景中,需要对手指运动等更精细的运动想象脑电图进行识别,以对MI-BCI系统进行进一步的推广。例如,对网络的结构进行调整和改进,结合图神经网络、3D神经网络等网络结构,捕捉更细微的数据变化;或者进行多模态融合,结合图像、肌电图等数据,利用不同信号之间的互补性提高精细分类的能力;或者借鉴EEG源分析技术,将不同脑区之间的交互模式作为分类的依据之一加以利用。

(2) 更稳定的泛化性能:尽管HA-FuseNet已经取得了一定的泛化性能力,但在BCI系统的实际应用中,各个脑电采集设备的硬件设施、目标用户的数据规模与数据分布往往具有差异性,因此仍需要进一步提高模型的泛化性能。例如,设计对不同时空分辨率、采样频率的数据具有自适应性的网络结构;在不同被试之间进行迁移学习,包括通过特定于EEG信号的数据对齐算法对不同被试的数据进行对齐,通过大量数据对模型进行预训练,再通过少量目标被试的数据进行微调等。

(3) 更通用的脑机接口:将模型的应用领域从运动想象扩展至更多的脑机接口范式,如稳态视觉诱发电位、事件相关电位、情绪识别等。例如,通过参数自适应算法使模型能够针对不同的任务自动地调整参数等。