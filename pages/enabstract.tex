% 英文摘要

Electroencephalogram(EEG)-based motor imagery classification emerges as a nascent field at the intersection of neuroscience and information science, holding promising applications in neurorehabilitation, human-machine collaboration, among others.  Currently, there are numerous methods for motor imagery EEG classification based on deep neural networks, yet several challenges persist: EEG signals are characterized by low signal-to-noise ratios and subject-specific variations, leading to pronounced performance variations among different subjects and suboptimal accuracy levels, necessitating models tailored to the unique characteristics of EEG data; existing methods are often constrained by factors such as spatial resolution, frequency filtering, and prior knowledge in neuroscience, compromising their stability in classification across diverse scenarios; the substantial parameter size of deep neural networks makes them prone to overfitting on small-scale datasets and increases computational costs. The thesis delves into these issues, with key contributions encompassing:

(1) To address the time-frequency-space characteristics of EEG signals and the constraints of existing methods, the thesis establishes two sub-networks based on CNN and LSTM networks, fusing feature maps extracted from diverse perspectives to achieve a more comprehensive and enriched representation of the time-frequency-spatial features, reducing reliance on specific conditions. A multiscale dense connection module is proposed, which designs convolutional kernels based on sampling frequency. This module enhances both the depth and breadth of feature extraction while integrating high-level semantic information with low-level features. Additionally, inverted bottleneck layers are incorporated into the model's architecture, facilitating the fusion of spatiotemporal features in the depth dimension. The strategy of focusing on temporal features is adopted to leverage the high temporal resolution of EEG signals more effectively, thereby diminishing the dependence on high spatial resolution.

(2) To address the low signal-to-noise ratio, non-stationarity, and subject specificity of EEG signals, the thesis proposes svSE hybrid attention module and SCoT global self-attention module to enhance the focus on critical data and mitigate the impact of artifacts and variability. Specifically, the svSE module improves representation of EEG's time-varying characteristics by introducing a variance layer and boosts adaptability to diverse data distributions through spatial and temporal axis-wise attention. The SCoT module integrates global contextual information from both spatial and temporal domains. It initially computes spatial self-attention based on axially projected feature maps, then incorporates local context into the calculation of spatiotemporal self-attention, calibrating feature maps and enhancing feature expression.

(3) To address the issues of large model parameter size, high computational cost, and the tendency to overfit on small datasets, the thesis presents a lightweight design for the model. It employs techniques such as axial convolutions and depthwise separable convolutions. Grounded in a thorough analysis of the merits and drawbacks of GhostNet, the SG lightweight convolution module is proposed. This module facilitates interaction among feature maps, thereby achieving a good balance between a reduced parameter count and maintaining high accuracy rates.

The thesis constructs an end-to-end motor imagery classification model, named HA-FuseNet, which is based on feature fusion and attention mechanisms. Extensive comparative experiments are conducted to ascertain the optimal architecture of the model. Experimental outcomes on the BCI Competition IV Dataset 2A illustrate that HA-FuseNet achieves average accuracies of 77.89\% and 68.53\% in subject-dependent and subject-independent experiments, respectively, with Kappa coefficients of agreement reaching 0.70 and 0.57, respectively. Compared to mainstream models such as EEGNet, HA-FuseNet yields an approximate average improvement of 8.42\% in subject-dependent accuracy and 9.26\% in subject-independent accuracy. Additionally, the thesis conducts generalization experiments with varying spatial resolutions on Dataset 2B, where HA-FuseNet attains mean accuracies of 75.23\% and 76.86\% in subject-dependent and subject-independent scenarios, respectively. These experimental findings validate the efficacy of the proposed model, thereby addressing prevailing issues in the field of motor imagery EEG classification to a considerable extent.