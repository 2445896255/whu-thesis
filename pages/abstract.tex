% 中文摘要

基于脑电图的运动想象分类是脑科学与信息科学交叉产生的新兴领域,能够实现大脑与外部设备之间的通信和控制,在神经康复、人机协作等领域具有良好的应用前景。目前基于深度神经网络的运动想象脑电图分类方法众多,但仍然存在以下问题:首先,脑电图信号具有低信噪比、被试特异性等特性,导致不同被试之间的性能差异明显,精度尚未达到理想水平,需要针对脑电图信号特性设计模型;其次,现有方法往往受到空间分辨率、频率滤波、神经科学先验知识等条件的制约,在不同场景下分类的稳定性有所欠缺;最后,深度神经网络较大的参数规模使得容易在小规模数据集上发生过拟合现象,且增加了计算成本。论文针对这些问题进行了深入研究,主要工作包括:

(1) 针对脑电图信号的时频空特性及现有方法受到制约的情况,分别基于卷积神经网络与长短期记忆网络搭建子网络,对从不同角度提取的特征图进行特征融合,以获得更全面丰富的时频空特征表达,降低对于特定条件的依赖性。提出了多尺度密集连接模块,该模块基于采样频率设计卷积核,在提升特征提取的深度和广度的同时,兼顾高级语义信息和低级特征;将反转瓶颈层引入网络中,在深度维度促进时空特征的融合;采取集中关注时间维度特征的策略,更好地利用脑电图信号的高时间分辨率,降低对于空间分辨率的依赖性。

(2) 针对脑电图信号的低信噪比、非平稳性和被试特异性,提出了svSE混合注意力模块和SCoT全局自注意力模块,提升模型对重要数据的关注度,降低伪迹和变化的干扰。svSE通过引入方差层更好地表征脑电图信号的时变特征,通过时空域的轴向注意力增强对不同数据分布的适应性。SCoT模块将时空域的全局上下文信息引入模型中,首先基于轴向投影的空间域特征图计算空间域自注意力,随后将局部上下文信息引入时空域自注意力的计算,对特征图进行校准并提升特征的表达能力。

(3) 针对网络参数规模大、计算开销高,小规模数据集上容易过拟合的问题,对模型进行了轻量化设计,采用轴向卷积、深度可分离卷积和下采样降低模型的参数量和计算量,并基于对GhostNet优点和缺点的深入分析,提出了SG轻量级卷积模块,该模块能够促进特征图之间的交互,从而在降低模型参数量的同时取得较好的准确率。

论文构建了一种基于特征融合和注意力机制的端到端运动想象分类模型HA-FuseNet,并进行多项对比实验确定模型的最优架构。在BCI Competition IV Dataset 2A数据集上的实验结果表明,HA-FuseNet在被试内和被试间实验中的平均准确率分别达到了77.89\%和68.53\%,Kappa一致性系数分别达到了0.70和0.57,较其他模型,被试内准确率的平均提升约为8.42\%,被试间准确率的平均提升约为9.26\%。论文同时在2B数据集上进行了不同空间分辨率的泛化性实验,实验结果表明HA-FuseNet在被试内和被试间实验的平均准确率分别达到了75.23\%和76.86\%。实验结果证明了论文所提出模型的有效性,在一定程度上解决了目前运动想象脑电图分类领域存在的问题。