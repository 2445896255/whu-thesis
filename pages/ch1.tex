% chapter1 绪论

\chapter{绪论}

\section{研究背景及意义}

大脑是一个结构和功能都很复杂的器官,是人体内外环境信息获得、存储、处理、加工及整合的中枢\cite{LXTX201902004},其神经活动蕴含着丰富的信息,直接反映了人类的思维、情绪和行为意图。
对大脑的研究,不仅可以防止大脑的衰退以及脑疾病的产生,而且可以通过模拟大脑促进人工智能的发展\cite{KYYX201907013}。

现今世界,各国政府及科研机构都高度重视脑科学研究的发展,美国于2013年启动“创新性神经技术大脑研究”计划,旨在开发和应用新的工具和技术,彻底改变人类对大脑的理解\cite{jorgenson2015brain}。
日本于2014年启动“脑/思维”计划,研究集中在三个领域:狨猴大脑的结构和功能图谱,用于脑图绘制的创新神经技术,以及人类脑图谱\cite{okano2015brain}。
欧盟于2013年启动“人类脑计划”,最初目标为通过超级计算机对人脑进行模拟,该项目已于2023年9月结束,其在神经科学领域取得了多项重大进展\cite{naddaf2023europe}。
中国于2021年将“脑科学与类脑研究”列为科技创新2030重大项目\cite{china2021brain},正式启动了“中国脑计划”,其内容包括认知神经机制的基础研究,脑疾病诊断和干预的转化研究以及脑启发智能(类脑)技术\cite{POO2016591}。

脑机接口(Brain Computer Interface,BCI)是脑科学与信息科学交叉产生的新兴学科领域,研究如何在大脑与外部设备之间建立直接的通信和控制通道,实现脑与设备之间的双向信息传输。
脑机接口系统的目标是将人类的思维、意愿或行动指令实时转化为可识别的控制信号,从而使人类可以直接由大脑而非神经肌肉通路来控制计算机或设备,
双向脑机接口系统不仅可以实现大脑控制,也为通过神经接口调节中枢神经系统提供了一种可能的方案\cite{he2020brain}。
脑机接口的基础在于采集大脑活动产生的生物信号,例如脑电图(Electroencephalography, EEG)、功能性磁共振成像(Functional Magnetic Resonance Imaging,fMRI)、近红外光谱(Near-Infrared Spectroscopy,NIRS)等,
这些生物信号从不同角度反映了大脑的认知过程、情绪状态和意图表达,其中,脑电图以其实时性、灵活性、便携性、低成本、非侵入性、高时间分辨率等优点,在脑机接口研究领域中受到了广泛的应用。

运动想象(Motor Imagery, MI)是脑机接口研究的主要方向之一,其表征的是一种运动意图,即个体在不实际执行物理动作的情况下,在大脑中想象执行特定动作的一种心理过程。
运动想象的研究源于对大脑功能区域的认知神经科学探索,研究发现,当人们想象执行某个动作时,即使身体并未实际做出该动作,大脑中的特定脑区仍会有所激活。
这种现象为基于运动想象的脑机接口系统提供了理论基础,通过识别和解码运动想象相关的脑电信号,可以基于使用者的自主意识实现对辅助设备的非侵入式控制,
% 其在改善脑卒中患者运动功能障碍\cite{ZKLS202103005},中、重度肢体运动功能障碍患者康复\cite{pichiorri2015brain},智能人机协作,
其在运动功能代偿、运动功能修复\cite{pichiorri2015brain}、智能人机协作、认知神经科学研究、游戏和虚拟现实等领域具有广阔的应用前景。

近年来,随着认知神经科学的持续发展以及各国政府对脑科学的日益重视,运动想象研究领域逐渐吸引了广泛的关注。
在基于脑电图信号进行运动想象分类的研究中,鉴于脑电信号固有的复杂性与异质性,传统机器学习方法通常需要依据神经科学领域的先验知识进行人工特征设计,
随着深度学习的迅速发展,研究者们开始将深度神经网络应用于运动想象分类任务中,期望能够自动化提取和学习脑电图信号中蕴含的潜在特征,
然而,受限于数据集规模、脑电数据质量、实时响应性能等因素,深度学习在该领域的应用效果尚未达到理想的水平。

综上所述,尽管运动想象研究已积累了一系列有价值的科研成果,但是依然存在一定的进步空间。
目前基于脑电图的运动想象分类任务仍然存在以下问题:

(1) 当前方法的实施常受限于多种条件,如特征提取依赖于专业神经科学知识与实践经验,对滤波的特定频率范围处理要求严苛,以及需要较高的电极采样密度。然而,在实际应用如家庭级和个性化的运动想象脑机接口系统中,往往不具备专家校验、标准滤波、高时空分辨率、高计算性能等理想条件,导致在实际场景的普适性上存在局限,应用场景较窄。

(2) EEG信号固有的非平稳性和被试特异性导致现有分类方法在不同个体间的性能表现差异较大,即使在部分被试上达到较高识别精度,但在其他被试上可能显著降低,这阻碍了运动想象分类模型在广泛人群中的稳定应用。因此,仍然需要进一步提升分类方法的精确度、稳健性和一致性,从而在各类被试中获得更为满意的表现,取得较为稳定均衡的性能。

(3) 基于深度学习的运动想象脑电图分类模型通常拥有庞大的参数规模,这在计算资源有限的边缘设备上运行效率低下,影响实时响应性能,从而制约了运动想象脑机接口系统的普及推广。因此,有必要研发对模型的参数量进行精简,以实现性能与效率的优化。

(4) EEG数据集因隐私保护、被试生理心理状态变化等因素,普遍具有规模偏小且样本多样性不足的特点。尽管数据增强是应对小规模数据集的有效策略,但针对具有被试特异性的EEG信号,某些增强方法可能破坏原始信号中蕴含的有价值信息。此外,数据增强过程会加重训练阶段的数据处理压力,延长训练周期,对运动想象脑机接口系统的实时响应性能有所制约。因此,如何在有限的小规模数据集上实现高效且稳定的性能提升,是需要考虑的问题。

针对这些问题,论文进行了

\section{国内外研究现状}

不同于常规的时间序列数据,运动想象脑电图数据具有独特的生理学特征,因此,对基于神经科学先验知识进行的特征提取研究进行回顾,
有利于运动想象领域的深度神经网络的设计。因此,本节将从脑电图特征提取和深度神经网络应用两方面来系统性地介绍运动想象领域的国内外研究现状。

\subsection{基于先验知识的运动想象脑电图特征研究现状}

传统机器学习中,对运动想象脑电图信号(MI-EEG)的处理通常分为预处理、特征提取和分类三个主要步骤\cite{altaheri2023deep},其中,分类算法与其它领域内广泛采用的传统机器学习技术类似,
因此,论文主要介绍与MI-EEG信号生理特性相关的预处理技术和特征提取方法的相关研究。

MI-EEG信号的信息密度在不同通道和频段间呈现出显著差异,并且数据的采集极易受到设备、环境、个体生理状态等因素的干扰,从而产生大量噪声,因此,对MI-EEG信号进行预处理是有必要的。
预处理步骤包括一系列操作,例如通道选择(为MI任务选择最有价值的EEG通道)、信号滤波(为MI工作选择最有意义的频率范围)、信号归一化(围绕时间轴对每个EEG通道进行归一化)、
伪影去除(从MI-EEG信号中去除噪声)\cite{altaheri2023deep}、基线矫正(消除EEG数据漂移带来的影响)等。
其中,伪影去除的经典方法是独立成分分析(Independent Component Analysis,ICA)和离散小波变换\cite{sai2017automated}。

MI-EEG信号的特征主要分为三类:时域特征、频域特征和空间域特征\cite{altaheri2023deep},此外,通过对原始数据的进一步加工和转换,可以构建一系列复合特征,如时频特征、时空特征、时频空特征等。

时域特征反映EEG信号随时间变化的特性,其包括均值、方差、标准差、峰度、偏度等统计量,此外,Bo Hjorth提出了一种快速计算时变信号的三个重要特征的方法,即活动性、移动性和复杂性,统称为Hjorth参数\cite{HJORTH1970306},
Luke等人使用卡尔曼滤波来处理EEG信号的不确定性\cite{7448410},考虑到EEG信号具有分形性质,Hsu提出了一种将分形维数和离散小波变换相结合的方法\cite{HSU2010295}。
空域特征反映EEG信号在不同脑电极上的分布情况,共空间模式(Common Spatial Pattern,CSP)是空域特征提取的经典方法,其核心思想是寻找一组最优的空间滤波器,
通过这组滤波器对原始多通道EEG数据进行空间投影,从而最大化两类信号在投影后空间的方差差\cite{wang2006common}。研究者对CSP算法进行了进一步优化,
提出了滤波器组共空间模式(FBCSP)\cite{ang2008filter}、共稀疏空间谱模式(CSSSP)\cite{dornhege2006combined}、共空间谱模式(CSSP)\cite{lemm2005spatio}、小波共空间模式(WCSP)\cite{mousavi2011wavelet}等方法,
在多分类、异步脑机接口等领域取得了更好的效果。
频域特征反映EEG信号在不同频带的功率,通常包括功率谱、功率谱密度、高阶谱、微分熵、傅里叶变换等方法。
Wang等人提出了一种基于快速傅里叶变换的功率谱特征提取方法\cite{wang2017tinnitus},Kroupi等人使用基于Welch方法的功率谱密度(PSD)估计来进行脑电情感分析\cite{kroupi2011eeg},
Herman等人比较分析了不同的频率特征在运动想象分类中的效果,并证实了功率谱密度具有最好的鲁棒性\cite{herman2008comparative}。
时频特征由于结合了时域特征和频域特征,在基于非平稳EEG信号的脑机接口研究中获得了广泛的使用\cite{pawar2020feature},
其主要包括匹配滤波、自回归模型、短时傅里叶变换、小波变换等方法,时频特征在提升了分类精度的同时,也提高了计算的复杂度。

尽管关于EEG信号特征提取的研究已积累了大量成果,然而传统预处理及特征提取方法仍存在着计算复杂度较高以及信息丢失的可能性,
同时,EEG信号的特征提取高度依赖神经科学领域的先验知识,从而限制了特征提取的自动化程度\cite{altaheri2023deep}。在这种背景下,深度学习被引入运动想象分类领域,
以克服传统方法存在的问题。

\subsection{基于深度学习的运动想象脑电图分类研究现状}

\section{研究内容}

\section{论文组织结构}