% chapter1 绪论

\chapter{绪论}

\section{研究背景及意义}

大脑是一个结构和功能都很复杂的器官,是人体内外环境信息获得、存储、处理、加工及整合的中枢\cite{LXTX201902004},其神经活动蕴含着丰富的信息,直接反映了人类的思维、情绪和行为意图。
对大脑的研究,不仅可以防止大脑的衰退以及脑疾病的产生,而且可以通过模拟大脑促进人工智能的发展\cite{KYYX201907013}。

现今世界,各国政府及科研机构都高度重视脑科学研究的发展,美国于2013年启动“创新性神经技术大脑研究”计划,旨在开发和应用新的工具和技术,彻底改变人类对大脑的理解\cite{jorgenson2015brain}。
日本于2014年启动“脑/思维”计划,研究集中在三个领域:狨猴大脑的结构和功能图谱,用于脑图绘制的创新神经技术,以及人类脑图谱\cite{okano2015brain}。
欧盟于2013年启动“人类脑计划”,最初目标为通过超级计算机对人脑进行模拟,该项目已于2023年9月结束,其在神经科学领域取得了多项重大进展\cite{naddaf2023europe}。
中国于2021年将“脑科学与类脑研究”列为科技创新2030重大项目\cite{china2021brain},正式启动了“中国脑计划”,其内容包括认知神经机制的基础研究,脑疾病诊断和干预的转化研究以及脑启发智能(类脑)技术\cite{POO2016591}。

脑机接口(Brain Computer Interface,BCI)是脑科学与信息科学交叉产生的新兴学科领域,研究如何在大脑与外部设备之间建立直接的通信和控制通道,实现脑与设备之间的双向信息传输\cite{DKJS202106005}。
脑机接口系统的目标是将人类的思维、意愿或行动指令实时转化为可识别的控制信号,从而使人类可以直接由大脑而非神经肌肉通路来控制计算机或设备,
双向脑机接口系统不仅可以实现大脑控制,也为通过神经接口调节中枢神经系统提供了一种可能的方案\cite{he2020brain}。
脑机接口的基础在于采集大脑活动产生的生物信号,例如脑电图(Electroencephalography, EEG)、功能性磁共振成像(Functional Magnetic Resonance Imaging,fMRI)、近红外光谱(Near Infrared Spectroscopy,NIRS)等,
这些生物信号从不同角度反映了大脑的认知过程、情绪状态和意图表达,其中,脑电图以其实时性、灵活性、便携性、低成本、非侵入性、高时间分辨率等优点,在脑机接口研究领域中受到了广泛的应用。

运动想象(Motor Imagery, MI)是脑机接口研究的主要方向之一,其表征的是一种运动意图,即个体在不实际执行物理动作的情况下,在大脑中想象执行特定动作的一种心理过程。
运动想象的研究源于对大脑功能区域的认知神经科学探索,研究发现,当人们想象执行某个动作时,即使身体并未实际做出该动作,大脑中的特定脑区仍会有所激活,这种生理现象使得基于运动想象的脑机接口系统成为可能。通过识别和解码运动想象相关的脑电信号,可以基于使用者的自主意识实现对辅助设备的非侵入式控制,
% 其在改善脑卒中患者运动功能障碍\cite{ZKLS202103005},中、重度肢体运动功能障碍患者康复\cite{pichiorri2015brain},智能人机协作,
其在运动功能代偿、运动功能修复\cite{pichiorri2015brain}、智能人机协作、认知神经科学研究、游戏和虚拟现实等领域具有广阔的应用前景。

近年来,随着认知神经科学的持续发展以及各国政府对脑科学的日益重视,运动想象研究领域逐渐吸引了广泛的关注。
在基于脑电图信号进行运动想象分类的研究中,鉴于脑电信号固有的复杂性与异质性,传统机器学习方法通常需要依据神经科学领域的先验知识进行人工特征设计,
随着深度学习的迅速发展,研究者们开始将深度神经网络应用于运动想象分类任务中,期望能够自动化提取和学习脑电图信号中蕴含的潜在特征,
然而,受限于数据集规模、脑电数据质量、实时响应性能等因素,深度学习在该领域的应用效果尚未达到理想的水平。

综上所述,尽管运动想象研究已积累了一系列有价值的科研成果,但是依然存在一定的进步空间。
目前基于脑电图的运动想象分类任务存在的问题包括但不限于:

(1) 当前方法的实施常受限于多种条件,如特征提取依赖于神经科学知识与实践经验,对滤波的特定频率范围处理要求严苛,以及需要较高的空间分辨率。然而,在实际应用如家庭级和个性化的运动想象脑机接口系统中,往往不具备专家校验、高时空分辨率等理想条件,导致在实际场景的普适性上存在局限,应用场景较窄。因此,有必要对端到端网络进行进一步设计研究,使其无需依赖人工特征提取与特定滤波,具备对不同空间分辨率的适应性;

(2) EEG信号固有的低信噪比、非平稳性和被试特异性导致现有分类方法在不同个体间的性能表现差异较大,且精度尚未完全达到理想水平,这阻碍了运动想象分类模型在广泛人群中的稳定应用。因此,仍然需要进一步提升分类方法的精确度和稳定性,从而在各类被试中获得更为满意的表现,取得较为稳定均衡的性能;

(3) 基于深度学习的运动想象脑电图分类模型通常拥有庞大的参数规模,同时,EEG数据集因隐私保护、被试生理心理状态变化等因素,普遍具有规模偏小的特点,使得往往需要进行数据增强,导致运行效率低下,影响实时响应性能,从而制约了运动想象脑机接口系统的普及推广。因此,有必要设计能够在小规模数据集上获取优良性能的网络,对模型的参数量进行精简,以实现性能与效率的优化。

为此,论文构建了一种端到端的运动想象脑电图分类网络HA-FuseNet,通过多尺度特征、深浅层特征融合、注意力机制、集中关注时间维度特征策略等方式提升特征提取的丰富度和完整度,并为不同重要程度的特征分配不同的注意力,从而实现分类精度的提升,并在不同的空间分辨率、不同被试之间取得良好的泛化性能。论文同时对HA-FuseNet进行了轻量化,以尽可能削减计算开销,获得更好的实时性性能。实验证明,HA-FuseNet在具有不同通道数量的小规模数据集的被试内和被试间实验中,均取得了较基准模型最优的性能表现,且未进行通道选择、频率滤波等预处理,证实了其作为端到端运动想象分类网络的有效性。

\section{国内外研究现状}

不同于常规的时间序列数据,运动想象脑电图数据具有独特的生理学特征,因此,对基于神经科学先验知识进行的特征提取研究的现状进行讨论,
有利于运动想象领域的深度神经网络的设计。因此,本节将从脑电图特征提取和深度神经网络应用两方面来系统性地介绍运动想象领域的国内外研究现状。

\subsection{基于先验知识的运动想象脑电图特征研究现状}

传统机器学习中,对运动想象脑电图信号(Motor Imagery Electroencephalogram,MI-EEG)的处理通常分为预处理、特征提取和分类三个主要步骤\cite{altaheri2023deep},其中,分类算法与其它领域内广泛采用的传统机器学习技术类似,
因此,论文主要介绍与MI-EEG信号生理特性相关的预处理技术和特征提取方法的相关研究。

MI-EEG信号的信息密度在不同通道和频段间呈现出显著差异,并且数据的采集极易受到设备、环境、个体生理状态等因素的干扰,从而产生大量噪声,因此,对MI-EEG信号进行预处理是有意义的。
预处理步骤包括一系列操作,例如选择对MI-EEG分类任务最有价值的通道、进行频率滤波以获取适当的EEG频带、对EEG信号进行归一化、
去除伪影噪声\cite{altaheri2023deep}、通过基线矫正消除EEG数据漂移带来的影响等。
其中,伪影去除的经典方法有独立成分分析(Independent Component Analysis,ICA)和离散小波变换\cite{sai2017automated}等。

MI-EEG信号的特征主要分为三类:时域特征、空域特征和频域特征\cite{altaheri2023deep},此外,通过对原始数据的进一步加工和转换,可以构建一系列复合特征,如时频特征、时空特征、时频空特征等。

时域特征反映EEG信号随时间变化的特性,其包括均值、方差、标准差、峰度、偏度等统计量,此外,Bo Hjorth提出了一种快速计算时变信号的三个重要特征的方法,即活动性、移动性和复杂性,统称为Hjorth参数\cite{HJORTH1970306},
Luke等人使用卡尔曼滤波来处理EEG信号的不确定性\cite{7448410},考虑到EEG信号具有分形性质,Hsu提出了一种将分形维数和离散小波变换相结合的方法\cite{HSU2010295}。
空域特征反映EEG信号在不同脑电极上的分布情况,共空间模式(Common Spatial Pattern,CSP)是空域特征提取的经典方法,其核心思想是寻找一组最优的空间滤波器,
通过这组滤波器对原始多通道EEG数据进行空间投影,从而最大化两类信号在投影后空间的方差差\cite{wang2006common}。研究者对CSP算法进行了进一步优化,
提出了滤波器组共空间模式(FBCSP)\cite{ang2008filter}、共稀疏空间谱模式(CSSSP)\cite{dornhege2006combined}、共空间谱模式(CSSP)\cite{lemm2005spatio}、小波共空间模式(WCSP)\cite{mousavi2011wavelet}等方法,
在多分类、异步脑机接口等领域取得了更好的效果。
频域特征反映EEG信号在不同频带的功率,通常包括傅里叶变换(Fourier Transform)、功率谱(Power Spectral)、高阶谱(Higher-order Spectral)、微分熵(Differential Entropy)、功率谱密度(Power Spectral Density,PSD)等方法。
Wang等人提出了一种基于快速傅里叶变换(Fast Fourier Transform,FFT)的功率谱特征提取方法\cite{wang2017tinnitus},Kroupi等人使用基于Welch方法的功率谱密度估计来进行脑电情感分析\cite{kroupi2011eeg},
Herman等人比较分析了不同的频率特征在运动想象分类中的效果,并证实了功率谱密度具有最好的鲁棒性\cite{herman2008comparative}。
时频特征将时域和频域的特征进行了结合,在基于非平稳EEG信号的脑机接口研究中获得了广泛的使用\cite{pawar2020feature},
其主要包括自回归模型(Autoregressive Model,AR)、匹配滤波(Matched Filtering,MF)、小波变换(Wavelet Transform,WT)、短时傅里叶变换(Short-time Fourier Transform,STFT)等方法,时频特征在提升了分类精度的同时,也提高了计算的复杂度。

尽管关于EEG信号特征提取的研究已积累了大量成果,然而传统预处理及特征提取方法仍存在着计算复杂度较高以及信息丢失的可能性,
同时,EEG信号的特征提取高度依赖神经科学领域的先验知识,从而限制了特征提取的自动化程度\cite{altaheri2023deep}。在这种背景下,深度学习被引入运动想象分类领域,
以克服传统方法存在的问题。

\subsection{基于深度学习的运动想象脑电图分类研究现状}

随着深度学习的迅速发展,基于深度学习的方法开始被应用于运动想象脑电图分类任务中,并取得了良好的效果。基于深度学习的运动想象脑电图分类方法主要包括三个部分:数据预处理,网络输入模式,以及深度神经网络架构。

在数据预处理阶段,由于与运动想象相关的脑电图事件相关去同步化和事件相关同步化两种生理现象主要发生在Mu节律(8-12Hz)和Beta节律(18-26Hz)中\cite{altaheri2023deep},多数深度学习方法采用带通滤波技术来提取相关频率成分。然而,这种方法可能会干扰或削减脑电图信号中的有效信息。此外,部分研究通过人工或自动手段去除脑电信号中的伪迹,例如Ma等人\cite{ma2019deep}采用了自动伪迹去除工具箱\cite{gomez2006automatic}。

在网络输入模式方面,研究\cite{luo2018exploring,olivas2019classification,she2019hierarchical,ma2020dwt,chu2018decoding,hassanpour2019novel}使用CSP、FBCSP、PSD、FFT、离散小波变换(Discrete Wavelet Transformation,DWT)等算法进行特征的提取,并将提取的特征作为深度神经网络的输入。然而,这些特征提取算法在一定程度上依赖于神经科学的先验知识和专家经验,并且人工设计的算法可能导致对某些重要信息的遗漏。Xu等人\cite{xu2018wavelet}使用WT算法将脑电图信号转化为频谱图,以更全面地表征时域和频域特征。Miao等人\cite{miao2020spatial}则使用FFT算法获取空频图,表征脑电图信号中的空域特征和频域特征。Li等人\cite{li2020novel}更进一步地将不同通道的数据进行组合,形成了时域、空域、频域的三维表示。尽管将脑电信号转换为图像形式有助于更好地表征EEG信号中不同域的信息,但这也会加剧计算资源的消耗,延长训练的时间。Amin\cite{amin2019deep}等人基于深度学习自动提取特征的概念,将EEG原始数据直接传入深度神经网络中进行训练。近年来,端到端的深度学习网络在运动想象脑电图分类任务中日益受到重视。

对于深度神经网络架构,卷积神经网络(Convolutional Neural Network, CNN)、循环神经网络(Recurrent Neural Network, RNN)等深度网络架构在运动想象脑电信号分类任务中被广泛应用。

卷积神经网络是深度学习中最常见的模型架构之一,擅长捕捉信号的局部特征,并随着网络层次加深逐步提取更高层次的抽象特征。在运动想象分类任务中,CNN不仅可以处理时频图、空频图和时空频三维图等转换后的数据,还可以直接应用于原始EEG数据。Schirrmeister等人\cite{schirrmeister2017deep}提出了一系列基于CNN的网络架构,包括ShallowConvNet和DeepConvNet,它们通过堆叠轴向卷积层替代传统的卷积层,能够在无需预先进行人工特征提取的情况下对脑电信号进行分类。Lawhern等人\cite{lawhern2018eegnet}提出了EEGNet,将深度卷积和可分离卷积引入网络架构中,构建了一种紧凑的通用型模型。Riyad等人\cite{riyad2021mi}在EEGNet的基础上提出了MI-EEGNet,针对运动想象分类任务进行了进一步的改进。Mane等人\cite{mane2021fbcnet}提出了FBCNet,将FBCSP算法的思想融入网络设计中,通过多个窄带滤波器对原始脑电图进行滤波,得到EEG信号的多视图表示。Zhang等人\cite{zhang2021eeg}使用了多个分支进行多尺度的特征提取,并通过残差连接加速网络的训练。Song等人\cite{song2022eeg}为了捕获脑电图信号中的长期依赖关系,将Transformer\cite{vaswani2017attention}模型引入CNN模型之后,直接处理经过时间和空间卷积层后的序列。Musallam等人\cite{musallam2021electroencephalography}基于残差神经网络(Residual Network,ResNet)\cite{he2016deep}构建了一种专用于运动想象脑电图分类的模型。Miao等人\cite{miao2023lmda}提出了一种专为脑电图信号解码设计的轻量级注意力模块,对ShallowConvNet和EEGNet进行了改进。

循环神经网络是一类专门设计用于处理序列数据的神经网络架构,它的特点是网络内部包含循环结构或反馈连接,使得信息可以在时间步之间传递和累积,特别适合处理具有时间依赖性或顺序结构的数据。RNN的变体有长短期记忆网络(Long Short Term Memory,LSTM),门控循环单元(Gated Recurrent Unit,GRU)等。LSTM模型是一种能够学习长期关系的RNN网络,克服了传统RNN的梯度消失问题,其在自然语言处理领域应用广泛,在运动想象任务中,LSTM也被用来处理EEG序列数据,获取其中蕴含的时域信息。Ma等人\cite{ma2018improving}使用滑动窗口对EEG原始数据进行截取从而进行数据增强,随后采用LSTM进行运动想象的分类。Wang等人\cite{wang2018lstm}提出了一种基于一维聚合近似(One Dimension-Aggregate Approximation,1d-AX)的LSTM模型执行运动想象脑电图分类任务。LKumar等人\cite{kumar2019brain}提出了一种使用CSP算法进行特征提取,线性判别分析进行特征约简,支持向量机作为分类器的LSTM模型,在GigaDB\cite{cho2017eeg}和BCI Competition IV-1\cite{blankertz2007non}数据集上分别取得了68.19\%和82.52\%的准确率。

基于深度神经网络强大的特征提取能力,现有方法已经在运动想象分类任务中积攒了相当的成果,然而,其实施常受限于多种条件,例如,深度神经网络往往具有较大规模的参数,不仅导致了运行效率低下,容易过拟合的问题,而且依赖于数据增强算法,进一步加剧了计算负担;由于EEG信号固有的非平稳性和被试特异性,一些方法在不同被试间的性能表现差异较大,即使在部分被试上达到较高识别精度,在其他被试上的性能也可能显著降低,在不同被试间的稳定性较差;一些方法依赖于较高的空间分辨率(电极采样密度)或特定频段的滤波,然而EEG信号的有用信息可能分布在较广范围的频率内,导致对数据相关性的破坏和有用信息的滤除,此外,在真实应用场景中,脑电采集设备可能不具备高空间分辨率等理想条件。基于以上问题,在运动想象脑电图分类领域,仍然有进行进一步研究的必要。

\section{研究内容}

针对上节所述运动想象脑电图分类领域存在的问题,论文结合EEG信号的特性和现有的深度神经网络进行研究和改进,旨在提出一种具有良好分类精度、泛化性能和计算效率的端到端模型。论文的研究内容主要包括以下方面:

(1) 对运动想象脑电图分类领域的现状及相关理论进行了研究和介绍。研究了运动想象相关的神经科学知识和脑电图信号的特性,为具有针对性的深度神经网络构建夯实理论基础,同时,介绍了运动想象脑电图分类领域的经典特征提取算法和深度神经网络,分析了这些方法的优势和不足,进一步阐述了当前仍然存在的问题。

(2) 运动想象相关的脑电活动模式比较固定,脑电信号的特征相对简单,因此高级语义信息与低级特征都需要被考虑进分类中,为了对不同尺度和层级的特征加以利用,论文在Inception结构的基础上引入了密集连接,从而形成多尺度密集连接模块(Dense Inception Module),通过基于采样频率设计的不同尺度的卷积核对不同尺度的时间/频率特征进行提取,并将不同层级的特征同时向下传递,以保留更全面的脑电图信号特征,提高模型的分类精度。为了丰富特征的表达,将反转瓶颈层引入网络中,从而在深度维度促进脑电图信号时空特征的融合。

(3) 脑电图信号具有较多的伪迹噪声,信噪比低,且随时间变化和被试变化具有非平稳性,因此,论文研究提出了一种混合注意力机制svSE模块,用以为脑电图信号中的重要部分分配更多的注意力,从而降低噪声和非平稳变化的干扰。svSE模块包含空间注意力机制和深度注意力机制,将方差信息引入计算之中,并基于轴向注意力对时间域和空间域的注意力分别进行计算,从而更好地感知运动想象过程中的时变特征,增强对不同数据分布的适应性,并降低计算开销。

(4) 为了获取脑电图信号中的长期依赖信息,论文研究提出了全局自注意力SCoT模块结合LSTM网络的方法,SCoT模块针对二维脑电图信号数据局部时空相关性较低的特点,首先通过轴向投影的方式计算空间域全局自注意力,随后,将局部上下文信息引入时空域的全局自注意力计算中,以提升特征的表达能力。通过两阶段的计算,SCoT模块在对全局自注意力进行校准的同时,减少了计算的开销。论文研究采用了深度维度的特征融合机制,从而对局部特征和全局特征同时加以利用,并避免特征之间的相互干扰。

(5) 为了在小规模数据集上取得较好性能,降低计算开销,论文对模型进行轻量化,采用轴向卷积和深度可分离卷积,并基于对GhostNet原理的分析,研究提出了一种相较于原始Ghost模块,能够更好地促进特征图之间交互的SG轻量级卷积模块,在降低参数规模和浮点运算数的同时,能够取得比原始Ghost模块更高的准确率。

(6) 经过一系列的改进,论文提出了基于特征融合和注意力机制的运动想象分类网络HA-FuseNet,并进行了一系列实验,展开了相应的分析和讨论。针对不同的网络结构设置进行了对比实验,对各个模块进行了消融实验,在BCI Competition IV Dataset 2A数据集和2B数据集上,基于同样的评估指标对基准模型和HA-FuseNet的被试内性能和被试间性能进行了对比实验。实验结果证明,论文提出的HA-FuseNet在运动想象脑电图分类任务上具有良好的分类精度、泛化性能和计算效率。

\section{论文组织结构}

论文主要分为以下五个章节:

第一章,绪论。首先介绍了运动想象脑电图分类研究的背景与意义,其次梳理了该领域的国内外研究进展,分别围绕基于神经科学先验知识的运动想象脑电图特征提取研究和基于深度学习方法的运动想象脑电图分类技术研究进行了介绍。最后对论文的研究目标、内容以及整体组织结构进行了介绍。

第二章,运动想象脑电图分类和深度神经网络基础。首先介绍了与运动想象脑电图分类任务相关的神经科学基础知识,其次介绍了卷积神经网络、循环神经网络、注意力机制等深度神经网络的相关基础知识。

第三章,基于特征融合和注意力机制的运动想象脑电图分类网络构建。针对现阶段运动想象脑电图分类任务中仍然存在的问题与脑电图信号的特性,基于Inception结构逐步进行改进,提出了多尺度密集连接、基于轴向注意力与方差层的混合注意力模块、基于轴向投影与上下文信息的全局自注意力模块、基于GhostNet改进的网络轻量化等,最终构建了基于特征融合和注意力机制的运动想象脑电图分类模型HA-FuseNet。对HA-FuseNet的结构和原理进行了详细的阐述,并介绍了其改进思路与构建过程。

第四章,实验结果分析与模型评估。介绍了实验所需的软硬件环境、所使用的运动想象脑电图数据集,以及运动想象脑电图分类任务的评价指标体系。其次介绍了运动想象脑电图数据处理方法,然后依次展开实验过程和结果讨论,包括消融实验、对比试验和跨被试泛化性实验等,验证了论文提出的方法在运动想象脑电图分类任务上的有效性。

第五章,总结与展望。总结论文的研究工作,分析论文所提方法的优点与局限性,并在此基础上,对可能的研究方向和改进策略进行了讨论。