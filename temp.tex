MI-EEG信号的采集是一个涉及多层面复杂交互的过程,其间采集的数据易受多种因素交织影响,其中包括但不限于硬件设备性能、受试者个体生理状态以及周围环境条件等变量。这些因素在数据的质量和特性上留下印记,具体体现在诸如通道信号分布的均匀性、伪影的产生与抑制等方面,使得有效解析和后续处理MI-EEG信号成为一项挑战性的工作。

\begin{enumerate}
    \item 通道选择:研究发现,运动想象分类的精度会随着通道数的逐渐增加而提高\cite{chang1998annual},但与此同时,
\end{enumerate}

轻量级:
+scse
+scse 激活函数由sigmoid换为softmax
+sese


eegtrans 等都经过了滤波预处理
在复现实验中效果不好
证明对噪声敏感


自注意力机制由Vaswani等人于2017年提出,是Transformer模型中的一种核心机制\cite{vaswani2017attention},
    其最早应用于自然语言处理领域,此后在计算机视觉等领域也得到了广泛的应用。
    自注意力机制允许神经网络在处理序列数据时,无需考虑输入序列的固定顺序或长度,
    而是通过计算每个位置上的元素与序列中所有其他元素的相关性来动态获取上下文信息,
    相较于其他注意力机制,自注意力机制减少了对外部信息的依赖,更擅长捕捉数据或特征的内部相关性。


    EEGNet\cite{lawhern2018eegnet}是一个用于EEG信号解码的紧凑的端到端网络,其采用了计算机视觉领域的深度卷积和可分离卷积,有效减少了模型的参数量。
EEGNet的结构如图~\ref{fig:EEGNet}所示,其通过三个模块对原始二维输入进行处理:
第一个模块包含时间卷积和空间卷积,其中,时间卷积通过多个卷积核将输入扩展到深度维度,空间卷积采用深度卷积以减少参数的数量;
第二个模块由包含深度卷积和逐点卷积的可分离卷积组成,在减少参数的同时,在深度维度上促进了时空特征的融合;
第三个模块用于分类,将特征直接传入softmax进行分类,从而减少自由参数的数量。
\begin{figure}
  \centering
  \includegraphics[width=\textwidth]{EEGNet.pdf}
  \caption{EEGNet结构}
  \label{fig:EEGNet}
\end{figure}

EEGNet模型简单、参数量小,在设计上借鉴了EEG信号解码领域的经典特征提取算法滤波器组共空间模式(Filter Bank Common Spatial Pattern,FBCSP),
非常适合小数据集的MI-EEG分类任务,是许多相关研究的基础网络。因此,论文以EEGNet为基础,进行下一步的改进。
