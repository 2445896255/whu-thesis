MI-EEG信号的采集是一个涉及多层面复杂交互的过程,其间采集的数据易受多种因素交织影响,其中包括但不限于硬件设备性能、受试者个体生理状态以及周围环境条件等变量。这些因素在数据的质量和特性上留下印记,具体体现在诸如通道信号分布的均匀性、伪影的产生与抑制等方面,使得有效解析和后续处理MI-EEG信号成为一项挑战性的工作。

\begin{enumerate}
    \item 通道选择:研究发现,运动想象分类的精度会随着通道数的逐渐增加而提高\cite{chang1998annual},但与此同时,
\end{enumerate}

轻量级:
+scse
+scse 激活函数由sigmoid换为softmax
+sese


eegtrans 等都经过了滤波预处理
在复现实验中效果不好
证明对噪声敏感


自注意力机制由Vaswani等人于2017年提出,是Transformer模型中的一种核心机制\cite{vaswani2017attention},
    其最早应用于自然语言处理领域,此后在计算机视觉等领域也得到了广泛的应用。
    自注意力机制允许神经网络在处理序列数据时,无需考虑输入序列的固定顺序或长度,
    而是通过计算每个位置上的元素与序列中所有其他元素的相关性来动态获取上下文信息,
    相较于其他注意力机制,自注意力机制减少了对外部信息的依赖,更擅长捕捉数据或特征的内部相关性。


    EEGNet\cite{lawhern2018eegnet}是一个用于EEG信号解码的紧凑的端到端网络,其采用了计算机视觉领域的深度卷积和可分离卷积,有效减少了模型的参数量。
EEGNet的结构如图~\ref{fig:EEGNet}所示,其通过三个模块对原始二维输入进行处理:
第一个模块包含时间卷积和空间卷积,其中,时间卷积通过多个卷积核将输入扩展到深度维度,空间卷积采用深度卷积以减少参数的数量;
第二个模块由包含深度卷积和逐点卷积的可分离卷积组成,在减少参数的同时,在深度维度上促进了时空特征的融合;
第三个模块用于分类,将特征直接传入softmax进行分类,从而减少自由参数的数量。
\begin{figure}
  \centering
  \includegraphics[width=\textwidth]{EEGNet.pdf}
  \caption{EEGNet结构}
  \label{fig:EEGNet}
\end{figure}

EEGNet模型简单、参数量小,在设计上借鉴了EEG信号解码领域的经典特征提取算法滤波器组共空间模式(Filter Bank Common Spatial Pattern,FBCSP),
非常适合小数据集的MI-EEG分类任务,是许多相关研究的基础网络。因此,论文以EEGNet为基础,进行下一步的改进。


Inception模块的层次数量和分支数量是影响其性能表现的两项关键参数,论文通过系统实验对这两项指标进行调整。参照文献\cite{santamaria2020eeg},实验中将Inception模块的层级结构分别设定为一层、二层和三层,分别命名为Inception-1、Inception-2和Inception-3。同时,基于实践经验与对EEG信号特征的理解,将分支数目分别设定为3、4和5个,相应地,将具有对应分支数量的Inception模块分别标记为Inception-n3、Inception-n4和Inception-n5,以探究不同层级与分支配置对模型性能的影响。在进行实验时,固定了除待调整参数之外的其他参数,实验结果分别如表~\ref{tab:inception-layer}~和表~\ref{tab:inception-block}~所示。
\begin{table}[ht]
  \centering
  \caption{Inception层次数量实验结果对比}
  \label{tab:inception-layer}
  \begin{tabularx}{\textwidth}{CCC}
    \toprule
    Models & ACC(\%) & Kappa \\
    \midrule
    Inception-1 & \textbf{75.71} & \textbf{0.67} \\
    Inception-2 & 72.57 & 0.63 \\
    Inception-3 & 66.85 & 0.56 \\
    \bottomrule
  \end{tabularx}
\end{table}
\begin{table}[ht]
  \centering
  \caption{Inception分支数量实验结果对比}
  \label{tab:inception-block}
  \begin{tabularx}{\textwidth}{CCC}
    \toprule
    Models & ACC(\%) & Kappa \\
    \midrule
    Inception-n3 & \textbf{72.54} & \textbf{0.63} \\
    Inception-n4 & 71.15 & 0.61 \\
    Inception-n5 & 71.42 & 0.62 \\
    \bottomrule
  \end{tabularx}
\end{table}

实验结果显示,采用单层Inception模块且分支数设定为3时,模型展现出了最优性能。随着Inception模块层数的递增,模型性能呈下降态势,这与之前研究认为浅层网络与特征相对简单的EEG信号更适配的观点相吻合。同时,随着参数规模的增大,过拟合风险也随之增加。

在探究分支数量变化的实验中,对于含3个分支的Inception模块,其特征提取的时间窗口设置为0.1秒、0.3秒和0.5秒,这对应着对4Hz及以上频带的信号特征提取。而对于拥有4个或5个分支的Inception模块,则在此基础上增加了0.7秒和0.9秒的时间窗口,从而进一步涵盖更低频段的特征提取。实验数据显示,随着分支数量增多,模型性能经历了先下降后回升的变化过程。这可能是由于2Hz以上的高频EEG信号(对应0.5秒以内的时间窗口)包含了更为关键的特征信息,虽然多尺度特征提取有助于提升网络性能,但较低频段的特征相对次要,而且,分支数量过多可能加剧过拟合风险,不利于模型的整体性能优化。

目前的大多数迁移都是做二分类 四分类比较少

奈奎斯特-香农(Nyquist–Shannon)采样定理指出,如果一个系统以超过信号最高频率至少两倍的速率对模拟信号进行均匀采样,那么原始模拟信号就能从采样产生的离散值中完全恢复


\begin{table}[ht]
  \centering
  \caption{轻量化卷积模块实验结果对比}
  \label{tab:lite}
  \begin{tabularx}{\textwidth}{CCCC}
    \toprule
    Models & Paramters & FLOPs & ACC(\%) \\
    \midrule
    GAS/SAS(0.5) & 7.31K & 130.37M & 74.61\\
    GAS/SAS(0.4) & 7.83K & 143.44M & 73.80\\
    GAS/SAS(0.6) & \textbf{6.80K} & \textbf{117.75M} & \textbf{74.61}\\
    SG(0.5) & 7.27K & 129.93M & 73.26\\
    SG(0.4) & 7.67K & 139.73M & 72.22\\
    SG(0.6) & 7.27K & 129.93M & 73.00\\
    Origin & 29.99K & 690.37M & \textbf{75.62}\\
    \bottomrule
  \end{tabularx}
\end{table}